\subsubsection{Catalan number}
\[ C_n = \frac{1}{n+1}{2n \choose n} \]
\[ C_{(n,m)} = {n+m \choose m} - {n+m \choose m-1} \]
\subsubsection{勾股数}
$n, m$互质且$m,n$至少有一个偶数则是苏勾股数
\[ a = m^2-n^2 \]
\[ b = 2mn \]
\[ c = m^2+n^2 \]
\subsubsection{容斥原理}
\[ g(A)=\sum_{S\,:\,S\subseteq A}f(S) \]
\[ f(A)=\sum_{S\,:\,S\subseteq A}(-1)^{\left|A\right|-\left|S\right|}g(S) \]
\subsubsection{Bell Number}
\[Bell_{n+1} = \sum\limits_{k=0}^n {n\choose k} Bell[k] \]
\[Bell_{p^m+n} \equiv mBell_{n} + Bell_{n+1} \pmod P\]
\[Bell_n = \sum\limits_{k=1}^{n} S(n,k) \]
\subsubsection{第一类Stirling数}
n个元素的项目分作k个环排列的方法数目
\[s(n+1,k)=s(n,k-1) + n \; s(n,k) \]
\subsubsection{第二类Stirling数}
第二类Stirling数是n个元素的集定义k个等价类的方法数目
\[S(n,k) = S(n-1,k-1) + k S(n-1,k)\]
\[S(n,k) =\frac{1}{k!}\sum_{j=1}^{k}(-1)^{k-j} {k\choose j} j^n \]
\subsubsection{判断是否为二次剩余}
若是d是p的二次剩余 $ d^{\frac{p-1}{2}} \equiv 1 \mod p $
否则 $ d^{\frac{p-1}{2}} \equiv -1 \mod p $
\subsubsection{级数展开式}
\begin{eqnarray*}
	e^x &=& \sum\limits_{n=0}^{\infty} \frac{x^n}{n!} \\
	log(1+x) &=& \sum\limits_{n=1}^{\infty} \frac{(-1)^{n+1}}{n}x^n (|x| < 1) \\
	\sin x &=& \sum\limits_{n=0}^{\infty} \frac{(-1)^n}{(2n+1)!}x^{2n+1} \\
	\cos x &=& \sum\limits_{n=0}^{\infty} \frac{(-1)^n}{(2n)!}x^{2n} \\
	\arcsin x &=& \sum\limits_{n=0}^{\infty} \frac{(2n)!}{4^n(n!)^2(2n+1)} x^{2n+1}
\end{eqnarray*} 
