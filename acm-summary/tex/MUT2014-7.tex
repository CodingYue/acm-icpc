\documentclass[a4paper, 11pt, nofonts, nocap, fancyhdr]{ctexart}
\usepackage{graphicx} 

\usepackage[margin=60pt]{geometry}

\setCJKmainfont[BoldFont={FZHei-B01}, ItalicFont={FZKai-Z03}]{FZShuSong-Z01}
\setCJKmonofont{FZShuSong-Z01}

\CTEXoptions[today=small]

\pagestyle{plain}



% \fancyhead[L]{\small{team name}}
% \fancyhead[C]{\small{FSTC 2014 - 05 - 训练报告}}
% \fancyhead[R]{\small{2014年8月2日}}

\renewcommand{\thesubsubsection}{Problem \Alph{subsubsection}.}
\newcommand{\problem}[1]{\subsubsection{#1}}

\title{Fudan ACM-ICPC Summer Training Camp 2014\\第10场训练报告}
\author{Team 1}
\date{\today}

\begin{document}

\maketitle

\section{训练过程}

(yy视角)

一开始我配环境, 邢皓明先看题, 然后发现G题有人过了, 于是xhm开始写题. 我从J题开始看题.

[G - 30min - 1Y]

我看了看H题, 觉得可以做, 让xhm写. 结果发现看错题了.

于是我先写E, 手贱wa了一发.

[E - 66min - 2Y]

xhm继续调H, 和我说了做法后让我写C. C题被卡了常数. 

[C - 120min - 3Y]

我们决定放弃H题. xhm开始写F. 又wa又T....

我之后推出了J题, 于是准备写J. 

[J - 190min - 1Y]

这时候场外观众lym接通了我们的场外求助热线, 把F题过了. 

[F - 212min - 8Y]

最后xhm写A, 我在纸上把D推清楚.

xhm写完A后, 发现TLE了. 于是换我写D. 

[D - 270min - 1Y]

最后我发现A题的正确做法, 和xhm说了之后改, 还是被卡常数了. 打算打表, 但是没有交上去.

比赛结束.

\section{解题报告}

\problem{Prime Tree}

\begin{description}
\item[负责] 邢皓明
\item[情况] 赛后通过
\end{description}

注意到答案只跟质因子的次数序列有关,爆搜一下发现本质上只有2958种输入(题目说数据是“几乎”随机的但是有<4000组不是随机的!简直恶意)。

于是对于每种输入只需要做一遍,$dp[i][j]$表示以$A_i$为根($A_i$是$n$的第$i$个约数),树高为j的概率,复杂度为$O(H \times ΣS(Ai))$,$H$为$n$的质因子个数,$S(x)$为$x$的约数个数。【注意压常数】

跑得有点慢,最终是打表通过的(常数写的小就不用打表)。

\problem{Rainbow Island}

\begin{description}
\item[负责] 邢皓明
\item[情况] 赛后通过
\end{description}

首先对于每个时刻的联通状态,我们只关心每个联通块的大小,于是本质上只有20的整数划分种方案($<700$种)

于是$f[i][j]$表示现在联通情况是第i种状态,人在j号节点,对于每个i,转移要解一个n元一次方程,复杂度$O(S \times n^3)$,$S$为状态数(极限数据$<700$)。【注意压常数】

\problem{Lucky Number}

\begin{description}
\item[负责] 杨越
\item[情况] 比赛中通过 - 120min - 3Y
\end{description}

大-小分治,对于结果是3位以上的数的情况,可行的进制$\leq 7000$ 枚举验证即可。

否则$3^4$枚举每位是什么,列一个二次方程$a\times base^2+b\times base+c=n$,解出base验证即可。

【注意压常数】

\problem{Seeing People}

\begin{description}
\item[负责] 杨越
\item[情况] 比赛中通过 - 270min - 1Y
\end{description}

对于一个同属于一个种类的人, 必然只有出发时间$t_i$相同时才能看到, 用树状数组维护.

对于不同的两个集合的人$i$, $j$. 假设他们$t$时刻相遇. 那么有$(t-ti)\times v1 = p_j$. 有$t = \frac{p_j}{v1}+t_i$. 那么此时
$j$所在的位置就是$(t-t_j)\times v2 = (\frac{p_j}{v1}+t_i)\times v2$ 看是否在区间$(x_i, x_i+w_i)$. 转化一下树状数组统计 

\problem{Stupid Tower Defense}

\begin{description}
\item[负责] 杨越
\item[情况] 比赛中通过 - 66min - 2Y
\end{description}

红塔放在后面肯定不会更劣
枚举后面放几个红塔,前面dp即可。

\problem{Destroy Transportation system}

\begin{description}
\item[负责] 刘炎明
\item[情况] 比赛中通过 - 212min - 8Y
\end{description}

实际上是一个01规划问题,对于一条边(u,v),如果u划成1,

\problem{Magical Forest}

\begin{description}
\item[负责] 邢皓明
\item[情况] 尚未通过
\end{description}

啥破题嘛!!

\problem{Game on Splay}

\begin{description}
\item[负责] 邢皓明
\item[情况] 比赛后通过
\end{description}

由于区间不会相交, 开个map暴力就好.

\problem{Spherical Surface}

\begin{description}
\item[负责] 邢皓明
\item[情况] 比赛后通过
\end{description}

\problem{Build the Park I}

\begin{description}
\item[负责] 杨越
\item[情况] 比赛后通过
\end{description}

\section{总结}

又是被卡常数啊. 好不爽. 读题还是要认真一些, 主要还是互相之间没法确认做法, 因为两人做比赛时间实在太紧迫了.

\end{document}

