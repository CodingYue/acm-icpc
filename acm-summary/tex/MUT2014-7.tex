\documentclass[a4paper, 11pt, nofonts, nocap, fancyhdr]{ctexart}
\usepackage{graphicx} 

\usepackage[margin=60pt]{geometry}

\setCJKmainfont[BoldFont={FZHei-B01}, ItalicFont={FZKai-Z03}]{FZShuSong-Z01}
\setCJKmonofont{FZShuSong-Z01}

\CTEXoptions[today=small]

\pagestyle{plain}



% \fancyhead[L]{\small{team name}}
% \fancyhead[C]{\small{FSTC 2014 - 05 - 训练报告}}
% \fancyhead[R]{\small{2014年8月2日}}

\renewcommand{\thesubsubsection}{Problem \Alph{subsubsection}.}
\newcommand{\problem}[1]{\subsubsection{#1}}

\title{Fudan ACM-ICPC Summer Training Camp 2014\\第10场训练报告}
\author{Team 1}
\date{\today}

\begin{document}

\maketitle

\section{概况}

本场训练,我们队伍在比赛中完成了6道题目,比赛后完成了3道题目,共完成9道题目。已经完成本场训练至少完成7题的要求

\section{训练过程}

(yy视角)

一开始我配环境, 邢皓明先看题, 然后发现G题有人过了, 于是xhm开始写题. 我从J题开始看题.

[G - 30min - 1Y]

我看了看H题, 觉得可以做, 让xhm写. 结果发现看错题了.

于是我先写E, 手贱wa了一发.

[E - 66min - 2Y]

xhm继续调H, 和我说了做法后让我写C. C题被卡了常数. 

[C - 120min - 3Y]

我们决定放弃H题. xhm开始写F. 又wa又T....

我之后推出了J题, 于是准备写J. 

[J - 190min - 1Y]

这时候场外观众lym接通了我们的场外求助热线, 把F题过了. 

[F - 212min - 8Y]

最后xhm写A, 我在纸上把D推清楚.

xhm写完A后, 发现TLE了. 于是换我写D. 

[D - 270min - 1Y]

最后我发现A题的正确做法, 和xhm说了之后改, 还是被卡常数了. 打算打表, 但是没有交上去.

比赛结束.

\section{解题报告}

\problem{Prime Tree}

\begin{description}
\item[负责] 邢皓明
\item[情况] 赛后通过
\end{description}

注意到答案只跟质因子的次数序列有关,爆搜一下发现本质上只有$2958$种输入(题目说数据是“几乎”随机的但是有$<4000$组不是随机的!简直恶意)。

于是对于每种输入只需要做一遍,$dp[i][j]$表示以$A_i$为根($A_i$是$n$的第$i$个约数),树高为j的概率,复杂度为$O(H \times ΣS(Ai))$,$H$为$n$的质因子个数,$S(x)$为$x$的约数个数。

【注意压常数】

跑得有点慢,最终是打表通过的(常数写的小就不用打表)。

\problem{Rainbow Island}

\begin{description}
\item[负责] 邢皓明
\item[情况] 赛后通过
\end{description}

首先对于每个时刻的联通状态,我们只关心每个联通块的大小,于是本质上只有20的整数划分种方案($<700$种)

于是$f[i][j]$表示现在联通情况是第$i$种状态,人在$j$号节点,对于每个$i$,转移要解一个$n$元一次方程,复杂度$O(S \times n^3)$,$S$为状态数(极限数据$<700$)。

【注意压常数】

\problem{Lucky Number}

\begin{description}
\item[负责] 杨越
\item[情况] 比赛中通过 - 120min - 3Y
\end{description}

大-小分治,对于结果是3位以上的数的情况,可行的进制$\leq 7000$ 枚举验证即可。

否则$3^4$枚举每位是什么,列一个二次方程$a\times base^2+b\times base+c=n$,解出base验证即可。

【注意压常数】

\problem{Seeing People}

\begin{description}
\item[负责] 杨越
\item[情况] 比赛中通过 - 270min - 1Y
\end{description}

简单统计问题,two-pointer扫描即可,终于不用压常数啦好开心

\problem{Stupid Tower Defense}

\begin{description}
\item[负责] 杨越
\item[情况] 比赛中通过 - 66min - 2Y
\end{description}

红塔放后面肯定不会更劣

枚举后面放几个红塔,前面dp即可。

\problem{Destroy Transportation system}

\begin{description}
\item[负责] 刘炎明
\item[情况] 比赛中通过 - 212min - 8Y
\end{description}

实际上是一个01规划问题,对于一条边$(u,v)$,如果$u$划成$0$,$v$划成$1$,答案会加上$B+D$,如果$u$是$1$,$v$是$0$,答案会减去$D$。

所以可以把$D$加到点权上$(val_u += D, val_v -= D)$,变成一个最大权闭合子图问题。

\problem{Magical Forest}

\begin{description}
\item[负责] 邢皓明
\item[情况] 比赛时通过 - 30min - 1Y
\end{description}

老梗,维护第i行现在是原来的第几行,第i列现在是原来的第几列就能回答询问了。

\problem{Game on S♂play}

\begin{description}
\item[负责] 邢皓明
\item[情况] 赛后通过
\end{description}

暴力修改+维护dfs序。

使用splay/treap会TLE,注意到旋转不会改变中序遍历的dfs序,所以可以静态维护区间积。

用线段树就能稳稳通过了。

\problem{K-th good string}

\begin{description}
\item[负责] 刘炎明
\item[情况] 尚未通过
\end{description}

写写写写写写写写写写写写写写写写写写写写题。

后缀数组,倍增,lcp,动态第k大,每步都是显然的,加起来就成了防ak题了。

\problem{FSF’s game}

\begin{description}
\item[负责] 杨越
\item[情况] 比赛时通过 - 190min - 1Y
\end{description}

$\sum \frac{A\times B}{gcd(A,B)}\times\sum\limits_{k|gcd(A,B)} k$

令$sumd_d = \sum\limits_{k|d} k$

考虑$A=N, B=i(i \leq n)$时的答案

$f_N = \sum\limits_{d|N} \sum\limits_{gcd(t, N) = d} t \times \frac{N}{d} \times sumd_d$

化简得 $f_N = \sum\limits_{d|N} sumd_d \times \frac{N}{d} \times sumd_d \times \sum\limits_{gcd(t, N) = d} t$

$f_N = \sum\limits_{d|N} sumd_d \times \frac{N}{d} \times sumd_d \times \sum\limits_{gcd(t, N/d) = 1} t \times d$

其中$\sum\limits_{gcd(t, N) = 1} t = \phi(N)*N/2$ 这是因为若$x$和$N$互质, 那么$N-x$也和$N$互质. 而与$N$互质的一共有$\phi(N)$ 个.

$f_N = \sum\limits_{d|N} sumd_d \times N \times sumd_d \times \phi(\frac{N}{d}) \times \frac{N}{d}$

所有东西都可以$Nlog(N)$预处理. 然后累加$f_N$作为答案即可

\section{总结}

又是被卡常数啊. 好不爽. 读题还是要认真一些, 主要还是互相之间没法确认做法, 因为两人做比赛时间实在太紧迫了.

\end{document}

