\documentclass[a4paper, 11pt, nofonts, nocap, fancyhdr]{ctexart}
\usepackage{graphicx} 

\usepackage[margin=60pt]{geometry}

\setCJKmainfont[BoldFont={FZHei-B01}, ItalicFont={FZKai-Z03}]{FZShuSong-Z01}
\setCJKmonofont{FZShuSong-Z01}

\CTEXoptions[today=small]

\pagestyle{plain}



% \fancyhead[L]{\small{team name}}
% \fancyhead[C]{\small{FSTC 2014 - 05 - 训练报告}}
% \fancyhead[R]{\small{2014年8月2日}}

\renewcommand{\thesubsubsection}{Problem \Alph{subsubsection}.}
\newcommand{\problem}[1]{\subsubsection{#1}}

\title{Fudan ACM-ICPC Summer Training Camp 2014\\第10场训练报告}
\author{Team 1}
\date{\today}

\begin{document}

\maketitle

\section{训练过程}

(yy视角)

一开始我配环境, 邢皓明先看题, 然后发现G题有人过了, 于是xhm开始写题. 我从J题开始看题.

[G - 30min - 1Y]

我看了看H题, 觉得可以做, 让xhm写. 结果发现看错题了.

于是我先写E, 手贱wa了一发.

[E - 66min - 2Y]

xhm继续调H, 和我说了做法后让我写C. C题被卡了常数. 

[C - 120min - 3Y]

我们决定放弃H题. xhm开始写F. 又wa又T....

我之后推出了J题, 于是准备写J. 

[J - 190min - 1Y]

这时候场外观众lym接通了我们的场外求助热线, 把F题过了. 

[F - 212min - 8Y]

最后xhm写A, 我在纸上把D推清楚.

xhm写完A后, 发现TLE了. 于是换我写D. 

[D - 270min - 1Y]

最后我发现A题的正确做法, 和xhm说了之后改, 还是被卡常数了. 打算打表, 但是没有交上去.

比赛结束.


\section{总结}

又是被卡常数啊. 好不爽. 读题还是要认真一些, 主要还是互相之间没法确认做法, 因为两人做比赛时间实在太紧迫了.

\end{document}

