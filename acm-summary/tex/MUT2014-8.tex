\documentclass[a4paper, 11pt, nofonts, nocap, fancyhdr]{ctexart}
\usepackage{graphicx} 

\usepackage[margin=60pt]{geometry}

\setCJKmainfont[BoldFont={FZHei-B01}, ItalicFont={FZKai-Z03}]{FZShuSong-Z01}
\setCJKmonofont{FZShuSong-Z01}

\CTEXoptions[today=small]

\pagestyle{plain}



% \fancyhead[L]{\small{team name}}
% \fancyhead[C]{\small{FSTC 2014 - 05 - 训练报告}}
% \fancyhead[R]{\small{2014年8月2日}}

\renewcommand{\thesubsubsection}{Problem \Alph{subsubsection}.}
\newcommand{\problem}[1]{\subsubsection{#1}}

\title{Fudan ACM-ICPC Summer Training Camp 2014\\第N场训练报告}
\author{Team 1}
\date{\today}

\begin{document}

\maketitle

\section{概况}

本场训练,我们队伍在比赛中完成了5道题目,比赛后完成了4道题目,共完成9道题目。已经完成本场训练至少完成9题的要求。

这里是其他概况。

\section{训练过程}

开场lym看到F是签到题上去秒了。没判a=b的情况re了一发。

[F - 21min - 2Y]

然后xhm上去写A,因为忽略了有可能输入的Ai不是2的幂,导致一直在wa,期间口胡了两发H,做法对了,因为姿势不正确没过。

[A - 126min - 4Y]

然后lym花了半小时证明H的做法没有任何问题,写了一个机智一点的根号算法就过了。

[H - 174min - 3Y]

yy写的B题徘徊在wa和tle的边缘,后来发现有一个double判>=0写成<=0了,于是改后过

[B - 240min - 5Y]

最后xhm上去口胡了个G的代码低空飞过。

[G - 257min - 2Y]

最后就过了5题..

\section{解题报告}

\problem{2048}

\begin{description}
\item[负责] 邢皓明
\item[情况] 比赛中通过 - 126min - 4Y
\end{description}

考虑f[i][j]表示考虑完了2^0,2^1...2^i之后,至少可以拼出j个2^i的方案数,答案是f[12][1],转移用组合数即可。

\problem{Area of Mushroom}

\begin{description}
\item[负责] 杨越
\item[情况] 比赛中通过 - 240min - 5Y
\end{description}

首先只有相等的才会影响且能走到最后. 于是对每个点, 将左右与其等速的点, 做一次极角排序, 如果某两个相邻的角度 $\geq \pi$那么就是infinity

\problem{GCD Array}

\begin{description}
\item[负责] 杨越
\item[情况] 比赛后通过
\end{description}

考虑一个一个操作$(n, d, v)$对一个询问$L$的影响, 答案是$v\times \sum\limits_{(t,n)=d\ t<=L} = v\times \sum\limits_{t|\frac{n}{d}} \lfloor\frac{L}{t\times d}\rfloor$

然后单独把$t\times d$拿出来, 令$x = t\times d$

$\sum\limits_{x=1}^{L} \lfloor \frac{L}{x} \rfloor f(x)$

考虑$f(x)$的意义, $x=t\times d, t | \frac{n}{d}$, 所以对于所有操作$(n, d, v)$ 对 $\frac{n}{d}$ 的所有约数$t$, 在$t\times d$ 打上加上加$v$的标记, 用树状数组维护前缀和. 最后由于只有根号锻 $\lfloor\frac{L}{x}\rfloor$是不同的, 所有单次复杂度均为$\sqrt{N}\times \log(N)$

\problem{Kingdom}

\begin{description}
\item[负责] 邢皓明
\item[情况] 赛后通过
\end{description}

首先问题一定是有解的。每次找出一个可以放在最后的点,对前面的方案没有影响,于是就把问题变成规模为n-1的子问题了。

\problem{Light}

\begin{description}
\item[负责] 邢皓明
\item[情况] 赛后通过
\end{description}

直接轮廓线dp,一共有3种状态:是0,是1,以及在这个位置使用了一个十字形修改。总状态O(n*m*2^m)。

\problem{Monster}

\begin{description}
\item[负责] 刘焱明
\item[情况] 比赛中通过 - 21min - 2Y
\end{description}

注意到提前休息没有意义,剩下情况显然。

\problem{Multiplication table}

\begin{description}
\item[负责] 邢皓明
\item[情况] 比赛中通过 - 257min - 2Y
\end{description}

凑出p-1之后,p-1+p-1就是p-2,p-1+p-2就是p-3,如此便可推出所有数。

\problem{Number Transformation}

\begin{description}
\item[负责] 刘焱明
\item[情况] 比赛中通过 - 174 - 3Y
\end{description}

不要瞎搞啊v\_v。

如果将每次变换后的数写成$a_i \times i$的形式,我们注意到$a_i$是单调减的,并且当$a_i \leq i$时$a_i$将不会再改变,而显然这个变换次数是根号级别的。下略。

\problem{Periodic Binary String}

\begin{description}
\item[负责] 无
\item[情况] 尚未通过
\end{description}

奇怪的题目。

\problem{Permanent}

\begin{description}
\item[负责] 无
\item[情况] 尚未通过
\end{description}

分治之后发现就是卷积。。求m问也还是卷积。。FFT就行了,不想写。。

\problem{Tree}

\begin{description}
\item[负责] 邢皓明
\item[情况] 赛后通过
\end{description}

建一个二分图,左边是非空的集合,右边是n个点,每个集合只能向不在其内部的点连边,最大匹配数即为方案数。
于是可以用dp解决。

\section{总结}

我是总结。

\end{document}

