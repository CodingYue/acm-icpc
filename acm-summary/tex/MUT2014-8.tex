\documentclass[a4paper, 11pt, nofonts, nocap, fancyhdr]{ctexart}
\usepackage{graphicx} 

\usepackage[margin=60pt]{geometry}

\setCJKmainfont[BoldFont={FZHei-B01}, ItalicFont={FZKai-Z03}]{FZShuSong-Z01}
\setCJKmonofont{FZShuSong-Z01}

\CTEXoptions[today=small]

\pagestyle{plain}



% \fancyhead[L]{\small{team name}}
% \fancyhead[C]{\small{FSTC 2014 - 05 - 训练报告}}
% \fancyhead[R]{\small{2014年8月2日}}

\renewcommand{\thesubsubsection}{Problem \Alph{subsubsection}.}
\newcommand{\problem}[1]{\subsubsection{#1}}

\title{Fudan ACM-ICPC Summer Training Camp 2014\\第N场训练报告}
\author{Team 1}
\date{\today}

\begin{document}

\maketitle

\section{概况}

本场训练,我们队伍在比赛中完成了N道题目,比赛后完成了N道题目,共完成N道题目。已经完成本场训练至少完成N题的要求(或,已经完成本场训练所有题目)。

这里是其他概况。

\section{训练过程}

我是过程。

\section{解题报告}

\problem{2048}

\begin{description}
\item[负责] 邢皓明
\item[情况] 比赛中通过 - 126min - 4Y
\end{description}

我是A题的报告。

\problem{Area of Mushroom}

\begin{description}
\item[负责] 杨越
\item[情况] 比赛中通过 - 240 - 5Y
\end{description}

我是B题的报告。

\problem{GCD Array}

\begin{description}
\item[负责] 杨越
\item[情况] 比赛后通过
\end{description}

我是C题的报告。

\problem{Kingdom}

\begin{description}
\item[负责] 邢皓明
\item[情况] 比赛后通过
\end{description}

我是D题的报告。

\problem{Light}

\begin{description}
\item[负责] 负责一、负责二
\item[情况] 尚未通过
\end{description}

我是E题的报告。

\problem{Monster}

\begin{description}
\item[负责] 负责一、负责二
\item[情况] 尚未通过
\end{description}

我是F题的报告。

\problem{Multiplication table}

\begin{description}
\item[负责] 邢皓明
\item[情况] 比赛中通过 - 257min - 2Y
\end{description}

我是G题的报告。


\problem{Number Transformation}

\begin{description}
\item[负责] 刘焱明
\item[情况] 比赛中通过 - 174 - 3Y
\end{description}

我是H题的报告。

\problem{Periodic Binary String}

\begin{description}
\item[负责] 负责一、负责二
\item[情况] 尚未通过
\end{description}

我是I题的报告。


\problem{Permanent}

\begin{description}
\item[负责] 负责一、负责二
\item[情况] 尚未通过
\end{description}

我是J题的报告。

\problem{Tree}

\begin{description}
\item[负责] 负责一、负责二
\item[情况] 尚未通过
\end{description}

我是K题的报告。


\problem{我是L题的标题}

\begin{description}
\item[负责] 负责一、负责二
\item[情况] 尚未通过
\end{description}

我是L题的报告。

\section{总结}

我是总结。

\end{document}

