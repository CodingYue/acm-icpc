\documentclass[a4paper, 11pt, nofonts, nocap, fancyhdr]{ctexart}
\usepackage{graphicx} 

\usepackage[margin=60pt]{geometry}

\setCJKmainfont[BoldFont={FZHei-B01}, ItalicFont={FZKai-Z03}]{FZShuSong-Z01}
\setCJKmonofont{FZShuSong-Z01}

\CTEXoptions[today=small]

\pagestyle{plain}



% \fancyhead[L]{\small{team name}}
% \fancyhead[C]{\small{FSTC 2014 - 05 - 训练报告}}
% \fancyhead[R]{\small{2014年8月2日}}

\renewcommand{\thesubsubsection}{Problem \Alph{subsubsection}.}
\newcommand{\problem}[1]{\subsubsection{#1}}

\title{Fudan ACM-ICPC Summer Training Camp 2014\\第N场训练报告}
\author{Team 1}
\date{\today}

\begin{document}

\maketitle

%\section{概况}

%本场训练,我们队伍在比赛中完成了N道题目,比赛后完成了N道题目,共完成N道题目。已经完成本场训练至少完成N题的要求(或,已经完成本场训练所有题目)。

%这里是其他概况。

\section{训练过程}

(YY视角)

上来lym配环境, 我和xhm看题.	xhm看了看E决定上去写

[49min - E - 1Y]

xhm写的过程中, 我们C, H, D都会做了. lym上去写C, 忘记输出空行wa一发

[53min - C - 2Y]

之后lym继续写D, 写完之后发现过不了样例, 就先下来调. xhm上去写G. 

[83min - G - 1Y]

然后我上去写H.

[103min - H - 1Y]

然后lym发现样例打错, 所以半天没试出答案. 那个输出本身就有问题.

[120min - D - 1Y]

xhm上去写B. 我准备写J. xhm B题写完wa. xhm自作聪明多判了条件. 我先开始写J. 然后xhm又发现可能有多条边, 会导致wa.

[163min - B - 1Y]

最后我开始一直waJ. 然后写完下来调, 让xhm写I, lym写F. 调了若干傻逼错误后

[270min - J - 7Y]

然后lym觉得写不完F, 于是让xhm去写I. 最后没过样例.

\section{总结}

打的还算顺利, 最后J题卡了很多发才过实在不该, 究其原因应该是对板子不够熟练. 开题顺序倒是还行, 虽然先写E, 但是做到了一发过. 

最后两人各卡一题.

\end{document}

