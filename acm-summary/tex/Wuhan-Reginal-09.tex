\documentclass[a4paper, 11pt, nofonts, nocap, fancyhdr]{ctexart}
\usepackage{graphicx} 

\usepackage[margin=60pt]{geometry}

\setCJKmainfont[BoldFont={FZHei-B01}, ItalicFont={FZKai-Z03}]{FZShuSong-Z01}
\setCJKmonofont{FZShuSong-Z01}

\CTEXoptions[today=small]

\pagestyle{plain}



% \fancyhead[L]{\small{team 1}}
% \fancyhead[C]{\small{FSTC 2014 - 05 - 训练报告}}
% \fancyhead[R]{\small{2014年8月2日}}

\renewcommand{\thesubsubsection}{Problem \Alph{subsubsection}.}
\newcommand{\problem}[1]{\subsubsection{#1}}

\title{Fudan ACM-ICPC Summer Training Camp 2014\\第12场训练报告}
\author{Team 1}
\date{\today}

\begin{document}

\maketitle

\section{概况}

本场训练,我们队伍在比赛中完成了7道题目,比赛后完成了2道题目,共完成9道题目。已经完成本场训练至少完成8题的要求

\section{训练过程}

(YY视角)

上来lym配环境, 我和xhm看题.	xhm看了看E决定上去写

[49min - E - 1Y]

xhm写的过程中, 我们C, H, D都会做了. lym上去写C, 忘记输出空行wa一发

[53min - C - 2Y]

之后lym继续写D, 写完之后发现过不了样例, 就先下来调. xhm上去写G. 

[83min - G - 1Y]

然后我上去写H.

[103min - H - 1Y]

然后lym发现样例打错, 所以半天没试出答案. 那个输出本身就有问题.

[120min - D - 1Y]

xhm上去写B. 我准备写J. xhm B题写完wa. xhm自作聪明多判了条件. 我先开始写J. 然后xhm又发现可能有多条边, 会导致wa.

[163min - B - 3Y]

最后我开始一直waJ. 然后写完下来调, 让xhm写I, lym写F. 调了若干傻逼错误后

[270min - J - 7Y]

然后lym觉得写不完F, 于是让xhm去写I. 最后没过样例.

\section{解题报告}

\problem{Assembling Services}

\begin{description}
\item[负责] 邢皓明
\item[情况] 比赛后通过
\end{description}

求每个程序的最早执行时间可以用Bellman-Ford迭代。

输出方案的话可以把每个点缀在最晚完成的前趋上,得到一棵树,然后输出这棵树的括号表示就是方案了。

(括号表示:比如一棵树是这样的,1的儿子有2,3,4,2的儿子有5,6,3的儿子有7,8,这棵树的括号表示就是1(2(5|6)|3(7|8)|4)

\problem{Box Relations}

\begin{description}
\item[负责] 邢皓明
\item[情况] 比赛中通过 - 163min - 3Y
\end{description}

首先xyz是独立的,可以转化成构造区间的问题,n个区间最多有2n个事件点,我们从左到右贪心地构造每个事件点。优先插入区间,没有可以插入的区间的话就选择一个已经满足所有相交条件的区间将其弹出,判断当前是否有能插入的区间用拓扑排序。

\problem{Crossing Rivers}

\begin{description}
\item[负责] 刘焱明
\item[情况] 比赛中通过 - 53min - 2Y
\end{description}

由于是随机的, 所以在人到达岸边的时候, 船的位置也是随机的, 故所有变量独立. 对各个河积分一下即可

\problem{Download Manager}

\begin{description}
\item[负责] 刘焱明
\item[情况] 比赛中通过 - 120min - 1Y
\end{description}

按题意直接模拟,或者注意到任意时刻带宽都是占满的直接算。

\problem{Exclusive-OR}

\begin{description}
\item[负责] 邢皓明
\item[情况] 比赛中通过 - 49min - 1Y
\end{description}

边权并查集,维护每个点i跟father[i]的异或值。询问有一两种特殊情况,不过都在样例里面。

\problem{Final Combat}

\begin{description}
\item[负责] 刘炎明
\item[情况] 尚未通过
\end{description}

按题意直接模拟,然后胡乱迭代加深胡乱剪枝,我还没过,因为实在不想碰。

\problem{Gift Hunting}

\begin{description}
\item[负责] 邢皓明
\item[情况] 比赛中通过 - 83min - 1Y
\end{description}

背包问题,直接dp即可。

\problem{Help Bubu}

\begin{description}
\item[负责] 杨越
\item[情况] 比赛中通过 - 103min - 1Y
\end{description}

只有8种数字, 然后压$2^8$mask表示有什么留在了序列中, 然后dp. 最后算那些所有被拿走的数字个数即可.

\problem{In A Crazy City}

\begin{description}
\item[负责] 邢皓明
\item[情况] 赛后通过
\end{description}

参考论文《Finding the Most Vital Node of a Shortest Path》。

\problem{Jiajia's Robot}

\begin{description}
\item[负责] 杨越
\item[情况] 比赛中通过 - 270min - 7Y
\end{description}

对于两个线段两两之间建一个圆, 然后圆并减去圆交. 因为任意两个圆的可行区域是他们的并减交. 然后最终可行区域是所有的并. 

\section{总结}

打的还算顺利, 最后J题卡了很多发才过实在不该, 究其原因应该是对板子不够熟练. 开题顺序倒是还行, 虽然先写E, 但是做到了一发过. 

最后两人各卡一题.

\end{document}

