\documentclass[a4paper, 11pt, nofonts, nocap, fancyhdr]{ctexart}
\usepackage{graphicx} 

\usepackage[margin=60pt]{geometry}

\setCJKmainfont[BoldFont={FZHei-B01}, ItalicFont={FZKai-Z03}]{FZShuSong-Z01}
\setCJKmonofont{FZShuSong-Z01}

\CTEXoptions[today=small]

\pagestyle{plain}



% \fancyhead[L]{\small{team name}}
% \fancyhead[C]{\small{FSTC 2014 - 05 - 训练报告}}
% \fancyhead[R]{\small{2014年8月2日}}

\renewcommand{\thesubsubsection}{Problem \Alph{subsubsection}.}
\newcommand{\problem}[1]{\subsubsection{#1}}

\title{Fudan ACM-ICPC Summer Training Camp 2014\\第12场训练报告}
\author{Team 1}
\date{\today}

\begin{document}

\maketitle

\section{概况}

本场训练,我们队伍在比赛中完成了7道题目,比赛后完成了N道题目,共完成N道题目。已经完成本场训练至少完成N题的要求(或,已经完成本场训练所有题目)。

\section{训练过程}

(YY视角)

上来lym配环境, 我和xhm看题.	xhm看了看E决定上去写

[49min - E - 1Y]

xhm写的过程中, 我们C, H, D都会做了. lym上去写C, 忘记输出空行wa一发

[53min - C - 2Y]

之后lym继续写D, 写完之后发现过不了样例, 就先下来调. xhm上去写G. 

[83min - G - 1Y]

然后我上去写H.

[103min - H - 1Y]

然后lym发现样例打错, 所以半天没试出答案. 那个输出本身就有问题.

[120min - D - 1Y]

xhm上去写B. 我准备写J. xhm B题写完wa. xhm自作聪明多判了条件. 我先开始写J. 然后xhm又发现可能有多条边, 会导致wa.

[163min - B - 3Y]

最后我开始一直waJ. 然后写完下来调, 让xhm写I, lym写F. 调了若干傻逼错误后

[270min - J - 7Y]

然后lym觉得写不完F, 于是让xhm去写I. 最后没过样例.

\section{解题报告}

\problem{我是A题的标题}

\begin{description}
\item[负责] 邢皓明
\item[情况] 比赛后通过
\end{description}


\problem{我是B题的标题}

\begin{description}
\item[负责] 邢皓明
\item[情况] 比赛中通过 - 163min - 3Y
\end{description}

我是B题的报告。

\problem{我是C题的标题}

\begin{description}
\item[负责] 刘焱明
\item[情况] 比赛中通过 - 53min - 2Y
\end{description}

由于是随机的, 所以在人到达岸边的时候, 船的位置也是随机的, 故所有变量独立. 对各个河积分一下即可

\problem{我是D题的标题}

\begin{description}
\item[负责] 刘焱明
\item[情况] 比赛中通过 - 120min - 1Y
\end{description}

按题意直接模拟,或者注意到任意时刻带宽都是占满的直接算。

\problem{我是E题的标题}

\begin{description}
\item[负责] 邢皓明
\item[情况] 比赛中通过 - 49min - 1Y
\end{description}

\problem{我是F题的标题}

\begin{description}
\item[负责] 刘炎明
\item[情况] 尚未通过
\end{description}

按题意直接模拟,然后胡乱迭代加深胡乱剪枝,我还没过,因为实在不想碰。

\problem{我是G题的标题}

\begin{description}
\item[负责] 邢皓明
\item[情况] 比赛中通过 - 83min - 1Y
\end{description}


\problem{我是H题的标题}

\begin{description}
\item[负责] 杨越
\item[情况] 比赛中通过 - 103min - 1Y
\end{description}

我是H题的报告。

\problem{我是I题的标题}

\begin{description}
\item[负责] 负责一、负责二
\item[情况] 尚未通过
\end{description}

我是I题的报告。

\problem{我是J题的标题}

\begin{description}
\item[负责] 杨越
\item[情况] 比赛中通过 - 270imin - 7Y
\end{description}

对于两个线段两两之间建一个圆, 然后圆并减去圆交. 因为任意两个圆的可行区域是他们的并减交. 然后最终可行区域是所有的并. 

\section{总结}

打的还算顺利, 最后J题卡了很多发才过实在不该, 究其原因应该是对板子不够熟练. 开题顺序倒是还行, 虽然先写E, 但是做到了一发过. 

最后两人各卡一题.

\end{document}

