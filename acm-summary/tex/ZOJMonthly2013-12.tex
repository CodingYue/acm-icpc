\documentclass[a4paper, 11pt, nofonts, nocap, fancyhdr]{ctexart}
\usepackage{graphicx} 

\usepackage[margin=60pt]{geometry}

\setCJKmainfont[BoldFont={FZHei-B01}, ItalicFont={FZKai-Z03}]{FZShuSong-Z01}
\setCJKmonofont{FZShuSong-Z01}

\CTEXoptions[today=small]

\pagestyle{plain}



% \fancyhead[L]{\small{team name}}
% \fancyhead[C]{\small{FSTC 2014 - 05 - 训练报告}}
% \fancyhead[R]{\small{2014年8月2日}}

\renewcommand{\thesubsubsection}{Problem \Alph{subsubsection}.}
\newcommand{\problem}[1]{\subsubsection{#1}}

\title{Fudan ACM-ICPC Summer Training Camp 2014\\第9场训练报告}
\author{Team 1}
\date{\today}

\begin{document}

\maketitle

\section{训练过程}

(YY 视角)

刚开场xhm从A开始看起, 我上机配环境. 配完xhm上来写B题, 我开始看题.

xhm写到一半, 我觉得I更为简单, 于是让xhm print代码我上去写I. 写完wa了. 看了看发现两人都把题目看错了. 于是让xhm继续写B. xhm交题, 发现TLE了. 然后我重新写I. 

之后xhm把memset去掉之后, A了B题. 

[38min - B - 4Y]

我继续写I. 然后[42min - J - 3Y]

之后写E. 然后写完又wa. 换xhm写F. 然后E一直wawawa, Fxhm写完了也wawawa. 于是我们决定先放一放E. 继续看别的题. D题我想了个$N^2log(N)$ 的做法, 觉得过不了. 于是先写G. 让xhm对拍F.

[149min - G - 1Y]

重新读E题, 发现漏看条件. 改了之后过了. 

[163 - E - 6Y]

邢皓明发现F题的trick. 改了之后过了

[168min - F - 4Y]

之后让xhm先试着写D题的$N^2log(N)$ 的做法, 写完之后TLE. 我C题和A题先后会做, 但是觉得好坑爹的题啊, C题要TLE的, A题讨论好麻烦. 于是帮xhm想D题的$N^2$做法, 告诉xhm后, 成功过了D。

[221 - D - 3Y]

最后我想硬着头皮写A, xhm发现J的做法. 我们俩一致觉得J题过的可能性最大. 于是让xhm写J.

写完之后TLE.................... sb题卡常数你麻痹, 赛后还发现数据和描述不符合.................................

比赛结束..

\section{总结}

作为打了这么多天的第一场逆风局, 发挥虽然欠妥, 但也和打了这么多场两个场有关系. 加上ZJU这帮傻逼出的题老是想卡人, 于是造成了低正确率. 感觉两人场没什么可以总结的. 看题还是需要更认真一些.

\end{document}

