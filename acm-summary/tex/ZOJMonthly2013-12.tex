documentclass[a4paper, 11pt, nofonts, nocap, fancyhdr]{ctexart}
\usepackage{graphicx} 

\usepackage[margin=60pt]{geometry}

\setCJKmainfont[BoldFont={FZHei-B01}, ItalicFont={FZKai-Z03}]{FZShuSong-Z01}
\setCJKmonofont{FZShuSong-Z01}

\CTEXoptions[today=small]

\pagestyle{plain}



% \fancyhead[L]{\small{team name}}
% \fancyhead[C]{\small{FSTC 2014 - 05 - 训练报告}}
% \fancyhead[R]{\small{2014年8月2日}}

\renewcommand{\thesubsubsection}{Problem \Alph{subsubsection}.}
\newcommand{\problem}[1]{\subsubsection{#1}}

\title{Fudan ACM-ICPC Summer Training Camp 2014\\第9场训练报告}
\author{Team 1}
\date{\today}

\begin{document}

\maketitle

\section{训练过程}

(YY 视角)

刚开场xhm从A开始看起, 我上机配环境. 配完xhm上来写B题, 我开始看题.

xhm写到一半, 我觉得I更为简单, 于是让xhm print代码我上去写I. 写完wa了. 看了看发现两人都把题目看错了. 于是让xhm继续写B. xhm交题, 发现TLE了. 然后我重新写I. 

之后xhm把memset去掉之后, A了B题. 

[38min - B - 4Y]

我继续写I. 然后[42min - J - 3Y]

之后写E. 然后写完又wa. 换xhm写F. 然后E一直wawawa, Fxhm写完了也wawawa. 于是我们决定先放一放E. 继续看别的题. D题我想了个$N^2log(N)$ 的做法, 觉得过不了. 于是先写G. 让xhm对拍F.

[149min - G - 1Y]

重新读E题, 发现漏看条件. 改了之后过了. 

[163 - E - 6Y]

邢皓明发现F题的trick. 改了之后过了

[168min - F - 4Y]

之后让xhm先试着写D题的$N^2log(N)$ 的做法, 写完之后TLE. 我C题和A题先后会做, 但是觉得好坑爹的题啊, C题要TLE的, A题讨论好麻烦. 于是帮xhm想D题的$N^2$做法, 告诉xhm后, 成功过了D。

[221 - D - 3Y]

最后我想硬着头皮写A, xhm发现J的做法. 我们俩一致觉得J题过的可能性最大. 于是让xhm写J.

写完之后TLE.................... sb题卡常数你麻痹, 赛后还发现数据和描述不符合.................................

比赛结束..

\section{解题报告}

\problem{Build the Park I}

\begin{description}
\item[负责] 杨越
\item[情况] 比赛后通过
\end{description}

对于每个小方块割成4个三角形.

分段考虑, 然后如果知道面积, 那么对于上下两部分套用台体的公式, 直接算出来.

在中间的部分, 需要分割成两个锥体. 那么答案是$\frac{(2*area_{down}+area_{up})\times h}{3}$

\problem{Buy the Pets}

\begin{description}
\item[负责] 邢皓明
\item[情况] 比赛中通过 - 38min(4Y)
\end{description}

我是B题的报告。

\problem{Cut the Cake}

\begin{description}
\item[负责] 负责一、负责二
\item[情况] 尚未通过
\end{description}

啥破题嘛!!

\problem{Water Level}

\begin{description}
\item[负责] 邢皓明
\item[情况] 比赛中通过 - 221min(3Y)
\end{description}

\problem{Eternal Reality}

\begin{description}
\item[负责] 杨越
\item[情况] 比赛中通过 - 163min(6Y)
\end{description}

$F_i$ 表示第$i$天的最大值. $F_i = F_{i-X-Y} + calc(i-X-Y+1, i)$ 

\problem{Bellywhite's Algorithm Homework}

\begin{description}
\item[负责] 邢皓明
\item[情况] 比赛中通过 - 168min(4Y)
\end{description}

我是F题的报告。

\problem{The Lambs}

\begin{description}
\item[负责] 杨越
\item[情况] 比赛中通过 - 149min(1Y)
\end{description}

对于任意两个stakes连有向边, 当且仅当所有的lambs在此边的左侧. 然后跑最小环

\problem{Tragedy Organ}

\begin{description}
\item[负责] 负责一、负责二
\item[情况] 尚未通过
\end{description}

我是H题的报告。

\problem{Salary Increasing}

\begin{description}
\item[负责] 杨越
\item[情况] 比赛中通过 - 42min(3Y)
\end{description}

由于区间不会相交, 开个map暴力就好.

\problem{Spherical Surface}

\begin{description}
\item[负责] 邢皓明
\item[情况] 比赛后通过
\end{description}

因为y的取值是整数, 所以只有180个取值范围. 然后枚举y1, 预处理y2所能跨越的长度, 然后树状数组回答.

\section{总结}

作为打了这么多天的第一场逆风局, 发挥虽然欠妥, 但也和打了这么多场两个场有关系. 加上ZJU这帮傻逼出的题老是想卡人, 于是造成了低正确率. 感觉两人场没什么可以总结的. 看题还是需要更认真一些.

\end{document}

