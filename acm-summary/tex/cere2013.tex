\documentclass[a4paper, 11pt, nofonts, nocap, fancyhdr]{ctexart}
\usepackage{graphicx} 

\usepackage[margin=60pt]{geometry}

\setCJKmainfont[BoldFont={FZHei-B01}, ItalicFont={FZKai-Z03}]{FZShuSong-Z01}
\setCJKmonofont{FZShuSong-Z01}

\CTEXoptions[today=small]

\pagestyle{plain}



% \fancyhead[L]{\small{team name}}
% \fancyhead[C]{\small{FSTC 2014 - 05 - 训练报告}}
% \fancyhead[R]{\small{2014年8月2日}}

\renewcommand{\thesubsubsection}{Problem \Alph{subsubsection}.}
\newcommand{\problem}[1]{\subsubsection{#1}}

\title{Fudan ACM-ICPC Summer Training Camp 2014\\第8场训练报告}
\author{Team 1}
\date{\today}

\begin{document}

\maketitle

\section{概况}

本场训练,我们队伍在比赛中完成了6道题目,比赛后完成了6题目,共完成12道题目。已经完成本场训练所有题目。

\section{训练过程}

yy视角

xhm从A看起, 我从L看起. 看完之后发现L是逗比题. 6min - L - 1Y.

然后xhm说B是逗比题, 于是让xhm写B. 19min - B - 1Y.

我看了J, 想了5分钟无果, 于是看I, 和xhm讨论之后发现是逗比题, xhm上去写了splay. 而我则去想C. 104min - I - 1Y.

C题想了错误算法, 写了一半发现不太对头. 于是去上了个厕所, 回来就发现是个逗比题. 于是 97min - C - 1Y.

xhm 开始写K, 我准备写D. xhm wa了一发K, 就print代码下来看. 我上去码D. 

xhm 觉得不可能wa, 然后把输出改成字典序最小. 就过了. 过了. 了...... (说好的spj呢?) 141min - K - 2Y.

D题真是恶心哭了. 想得太简单, 以为是个最短路.  期间xhm 上去写了F. 191min - F - 1Y. 然后xhm说会写E了, 结果他也不太精神. 就没过E.

比赛结束.

\section{解题报告}

\problem{Rubik’s Rectangle}

\begin{description}
\item[负责] 杨越
\item[情况] 赛后通过
\end{description}

一看到这题就想到了魔方,  如何在不影响别的情况下做移动 。

先考虑必然无解的情况。很容易想到对中心点上下左右对称的可以看做一个东西. 如果对于原矩阵来说, 4个角的某个数不在这四个里面必定无解.

然后就考虑如何不影响别的. 参考了以前玩魔方的经历, 存在着一种不影响4个以外的格子的操作. 效果是将某个直角的三个顶点, 做一次shift。 可以shift left 亦可以shift right.

有了这个的话, 我们可以把 1移到左上角, 4移到右下角. 这时如果2, 3已经排好, 就行了. 如果排不好, 那么就进入到下一种情况.

我们称这个是特殊的. 那么我们可以对某一行或者某一列翻转奇数次. 只要翻转一次, 特殊就变成不特殊, 否则不特殊变成特殊. 然后对于每一个, 我们可以规约到左上角. 也就是, 可以通过左上角的行列操作, 实现交换2, 3位置.

然后就可以想到这么一个模型, 是否存在一种方案, 对于某一行或者某一列flip, 然后将矩阵内所有为1的点变成0. 这个可以二分图染色.

于是算法就是求出一个方案, 让所有的各点都是非特殊的, 然后在shift成我们所需要的矩阵.

\problem{What does the fox say?}

\begin{description}
\item[负责] 邢皓明
\item[情况] 比赛中通过 - 19min - 1Y 
\end{description}

开场题.

\problem{Magical GCD}

\begin{description}
\item[负责] 杨越
\item[情况] 比赛中通过 - 104min - 1Y
\end{description}

枚举右端点后, 考虑决策点. 对于从右开始的一段整体gcd相等的, 合并成一段. 然后发现由于gcd递减, 所以只会有log个不同的gcd值. 暴力即可.

\problem{Subway}

\begin{description}
\item[负责] 杨越
\item[情况] 赛后通过
\end{description}

对于链bfs, 然后由于每条链只会经过一次, 于是对于链上的dp一下就好了.

\problem{Escape}

\begin{description}
\item[负责] 邢皓明
\item[情况] 赛后通过
\end{description}

先把1到t的链拿出来,形成一个主链+若干子树的结构。

对于一个子树,如果它是一条链,我们可以将这条连“规整”成正负相间的样子。

令一负一正为一对,对于相邻的两对,如果前面那对的收益$\leq0$,则合并到后面那对,如果前面那对的负值的绝对值大于后面的负值的绝对值,则同样合并两对经过这样“规整”后,变成了收益都$>0$,按负值单增的一条链。

如果对于一个点,它的所有儿子都是这样的链,则我们可以通过类似归并排序的手段(实现中使用启发式合并或者可并堆)将其弄成一条这样的链,因而我们递归地证明了任何一棵子树都可以规整成这样一条链。

然后就是简单的贪心问题了。

\problem{Draughts}

\begin{description}
\item[负责] 邢皓明
\item[情况] 比赛中通过 - 191min - 1Y
\end{description}

签到题. 爆搜.

\problem{History course}

\begin{description}
\item[负责] 邢皓明
\item[情况] 赛后通过
\end{description}

先二分答案,变成check是否存在一种方案使得相交的区间的位置间隔不超过$k$的子问题。

这个子问题是可以贪心的,具体思路:每次选择能选的区间中右端点尽量靠左的区间,判断是否能选可以动态维护对于每个还没加入的区间,至少要在哪个位置前加入,笔者用了两棵线段树实现,总复杂度$O(n * log^2n)$。

\problem{Chain \& Co}

\begin{description}
\item[负责] 邢皓明
\item[情况] 赛后通过
\end{description}

可以按方向将正方形分成X Y Z三个集合,因为平行的正方形都是separable的,所以答案肯定是$X + \{Y,Z\}$或者Y + $\{Z,X\}$或者Z + $\{X,Y\}$。

判断合法可以做一个转化,两个长方形是inseparable的当且仅当其中一个的最高点在另一个长方形所张成的一个$2r \times r \times r$的立方体中,一个点在所有立方体等价于在所有立方体的交中,所以求出每对集合与集合的判断,求出其中一个集合中所有立方体的交,就能够对另一个集合中的每个正方形实现$O(1)$判断了,复杂度$O(n)$。


\problem{Crane}

\begin{description}
\item[负责] 邢皓明
\item[情况] 比赛中通过 - 104min - 1Y
\end{description}

这个可以每次把某个值移到最终位置, 只需要 $log N$ 步以内. 然后splay维护。 

\problem{Captain Obvious and the Rabbit-Man}

\begin{description}
\item[负责] 杨越
\item[情况] 赛后通过
\end{description}

把$p_n$ 看成通项, 那么$Fib_k$就是特征根. 于是特征多项式就是 $\Pi (x-Fib_k)$

之后递推就好了

\problem{Digraphs}

\begin{description}
\item[负责] 邢皓明
\item[情况] 比赛中通过 - 141 - 2Y
\end{description}

把禁止的二元组看成在完全图删去一些边,找一条最长路,令最长路经过的点为$S[L]$,则可以构造出一个规模为$\frac{L+1}{2}$的矩阵,令$A[i][j] = S[i + j - 1]$即可,可以证明这是答案的上界。

\problem{Bus}

\begin{description}
\item[负责] 杨越
\item[情况] 比赛中通过 - 6min - 1Y
\end{description}

答案是$2^n-1$

\section{总结}

打的真是累哭了. 少一个人两个人基本没有讨论的时间. 比如A题, 如果两个人讨论讨论一定能做出来. 想题的时间也不够. 

\end{document}

