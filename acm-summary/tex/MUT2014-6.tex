\documentclass[a4paper, 11pt, nofonts, nocap, fancyhdr]{ctexart}
\usepackage{graphicx} 

\usepackage[margin=60pt]{geometry}

\setCJKmainfont[BoldFont={FZHei-B01}, ItalicFont={FZKai-Z03}]{FZShuSong-Z01}
\setCJKmonofont{FZShuSong-Z01}

\CTEXoptions[today=small]

\pagestyle{plain}



% \fancyhead[L]{\small{team 1}}
% \fancyhead[C]{\small{FSTC 2014 - 05 - 训练报告}}
% \fancyhead[R]{\small{2014年8月2日}}

\renewcommand{\thesubsubsection}{Problem \Alph{subsubsection}.}
\newcommand{\problem}[1]{\subsubsection{#1}}

\title{Fudan ACM-ICPC Summer Training Camp 2014\\第7场训练报告}
\author{Team 1}
\date{\today}

\begin{document}

\maketitle

\section{概况}

本场训练,我们队伍在比赛中完成了6道题目,比赛后完成了4道题目,共完成10道题目。已经完成本场训练所有题目。

\section{训练过程}

比赛开始时非常坑爹,题目过了十几分钟才打印好,纳指导急中生智让大家用自己的电子设备看题。。于是比赛乱哄哄地开始了。我从前往后看题,yy先上去配环境。

这一场比赛题面很长,花了几分钟读懂了A的题意,留意到跟hdu上数据范围不太一致,纳指导表示以oj上描述为准。

只有一个人看题效率太低了,于是刷了刷board,发现大家在交E题和G题,遂开始看E。

E好像是个贪心,按黑白染色种树应该是最优的,不太确定做法正确性,但已经有两位数的AC了,又看了一下G题感觉很水,扔给yy之后上机写E。\

写完E之后wa了。。看到通过/提交的数字是70+/120+,感觉很不高兴,盯着代码看了好久发现n=m=1的时候会错。。sad,遂2Y。

[21min E 2Y]

通过E后yy表示G题应该就是杨辉三角,但是是正负交替的,需要高精度,确认了下做法没啥问题就让他拍java了,过样例1Y,期间交错题结果给A贡献了一次错误提交。

[41min G 1Y]

继续刷board,发现C和J陆续有人过,C是一个很像数学题的题目,想起一道类似的题目,不过是整数的,是一个单调栈+可并堆的做法,想了想好像可以套用在这道题上,还不需要开堆,说服yy之后我上去写C,yy去看其他题目。(为什么不做J呢。。因为正常节奏J早就被lym秒掉了。。sad),一会C样例过了,提交AC。

[93min C 1Y]

yy看了一坨题之后表示还是J好做,于是上去写J,我思考A的细节准备yy下来后立刻上去写。

J题写完过了样例,打印下来代码看逻辑,我先上去写A。看了一会yy表示代码没有问题,提交AC。

[123min J 1Y]

A写完样例死活算不对,手算跟程序输出一样,board上也没有新的题目有人通过,十几分钟后才留意到输入是从0开始编号的,于是终于算对了样例,提交AC。

[162min A 2Y(第一发是G题交错题)]

然后我和yy各选了一个坑跳进去,我去写D,他写F。

D题写完TLE,加了个剪枝WA,换yy写F。

D题改正了几个手贱之后还是WA,F题写完提交TLE。

我开了一下脑洞,猜测D题输入数据中选手喜欢/讨厌自己的情况,而我的程序对于这种情况会把边权算错,改了一下提交发现过了,好无聊。。

[274min D 5Y]

F题优化了好几个做法还是TLE,试了试极限数据发现本地要跑7.9秒,而时限7秒,于是我们试图二分出5组数据中的某一组的答案然后骗过去。。没有来得及二分出来比赛就结束了。

最后6题,凭借罚时优势排到了rank3,感觉这场真是恶心,把大家都恶心死了。 

\section{解题报告}

\problem{Map}

\begin{description}
\item[负责] 邢皓明
\item[情况] 比赛中通过 - 182min - 2Y
\end{description}

注意到给定图是不超过$10$条链构成,我们可以枚举每个位置然后$2^{10}$计算出所有链上这个位置的边对答案的贡献。

\problem{Hello, Your Package!}

\begin{description}
\item[负责] 邢皓明
\item[情况] 比赛后通过
\end{description}

预处理road与road的交点,road到package的最近点。。作为关键点,然后求出两两间的最短路,然后利用最短路的值算出搭车的最优方案,然后问题变成了最短哈密顿路径问题,再做一次$O(n \times 2^n)$的状压dp就可以了。

\problem{Room and Moor}

\begin{description}
\item[负责] 邢皓明
\item[情况] 比赛中通过 - 103min - 1Y
\end{description}

与BOI2004 数字序列类似的做法,维护每一段的最优决策,如果当前段的最优决策比前面段的决策值要小,则与前面的段合并,再求最优决策。。从而用一个类似单调栈的思路求出所有B[i],于是解决本题。

\problem{Football Manager}

\begin{description}
\item[负责] 邢皓明
\item[情况] 比赛中通过 - 274min - 4Y
\end{description}

枚举所有$\choose{20}{11}$种人员方案(不在意具体分配),然后用一个$f[a][b][c][d]$的dp就可以了,需要剪枝,一个比较弱但是能AC的剪枝是,如果所有人踢最擅长的位置(不管阵容如何)也无法比当前答案更优,放弃这个解,不进行dp,直接枚举下一组,从而卡过本题。

\problem{Apple Tree}

\begin{description}
\item[负责] 邢皓明
\item[情况] 比赛中通过 - 27min - 2Y
\end{description}

$(n=m=1)$时答案为$1$,否则在所有$i+j$是偶数的$(i,j)$施肥即可。

\problem{Robbery Plan}

\begin{description}
\item[负责] 邢皓明
\item[情况] 比赛后通过
\end{description}

首先有$(x / a) mod b = (x mod (a \times b) / a)$ mod b
所以对于每个p求出$f[p,d] mod ((p + 1) \times M)$的值即可。
注意到$f[p,d] = \sum \choose{p - 1}{k} \times f[1, d - k \times e]$
预处理组合数,对每个模做一下就可以求出best数组,然后就是简单的背包了。

\problem{Series 1}

\begin{description}
\item[负责] 杨越
\item[情况] 比赛中通过 - 41min - 1Y
\end{description}

推一推之后发现是组合数, 然后用java写就行了, 用 $O(N)$来边循环边求组合数.

\problem{Series 2}

\begin{description}
\item[负责] 杨越
\item[情况] 比赛后通过
\end{description}

只会做$log(N)$步, 所以暴力就好了.

\problem{Another Letter Tree}

\begin{description}
\item[负责] 邢皓明
\item[情况] 比赛后通过
\end{description}

预处理每个点到根的路径上有多少个子序列是$s[l,r]$,这需要$O(N \times L^2)$的时间。
询问的时候利用x,y,lca(x,y)的信息容斥一下就可以把答案算出来了。

\problem{Fighting the Landlords}

\begin{description}
\item[负责] 杨越
\item[情况] 比赛中通过 - 123min - 1Y
\end{description}

模拟题, 题说什么写什么. 注意一下如果可以直接跑完就是赢. 还有炸弹的比较. 如果你没炸弹又不能走完是输的.

\section{总结}

好恶心的题啊. 深切的感受到没法想题的痛苦. $1 + 1 + 1 > 3$, 但是 $1+1 = 2$. 

\end{document}

