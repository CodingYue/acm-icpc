\documentclass[a4paper, 11pt, nofonts, nocap, fancyhdr]{ctexart}
\usepackage{graphicx} 

\usepackage[margin=60pt]{geometry}

\setCJKmainfont[BoldFont={FZHei-B01}, ItalicFont={FZKai-Z03}]{FZShuSong-Z01}
\setCJKmonofont{FZShuSong-Z01}

\CTEXoptions[today=small]

\pagestyle{plain}

% \fancyhead[L]{\small{team name}}
% \fancyhead[C]{\small{FSTC 2014 - 05 - 训练报告}}
% \fancyhead[R]{\small{2014年8月2日}}

\renewcommand{\thesubsubsection}{Problem \Alph{subsubsection}.}
\newcommand{\problem}[1]{\subsubsection{#1}}

\title{2013上海邀请赛总结}
\author{Team 1}
\date{\today}

\begin{document}

\maketitle


\section{比赛过程}

(YY视角)

上来分头看题, xhm发现A是逗题, 于是写完交. 手贱了两发. A - 11min - 3Y.

我发现F是傻逼题, 然后lym开始写F, 写完交, F - 24min - 1Y. 

lym继续写J. 然后莫名wa了, 下来调.

我开始写I. 太久没打比赛了, 都忘记确认题意. 写完wa.

这时lym开了脑洞, 觉得这非等宽字体导致空格被吞. 于是改了交 J - 45min - 3Y.

lym 顺便把D做了. D - 131min - 2Y.

然后xhm发现C是原题, 上去写网络流. C - 153min - 2Y.

这时我改了交, 竟然过了= =. 可是后来被rejudge称wa了.

lym上去写B。 B - 176min - 1Y.

此时lym开始写G. 然后交了wa. 我们感到不可思议.

于是写E, 我看G的代码. E - 216 - 2Y.

最后我们看H题, 一致觉得只能暴力. 于是交了暴力. H - 259min - 1Y.

最后到死都过不了G. 因为出题人也不过不了样例.



\section{总结}

作为恢复状态的第一场, 简直被题目气哭了. 非等宽字体打印, rejudge成wa, G题到死都过不掉! 

暴露了一些问题. 这一次因为J题卡了一下, 然后导致没有人与yy一起确认I题的题意, 导致了I题的后续wa. 

之后无论有没有卡题, 一定要确定题意. 

还有就是卡题之后不要急躁, 不要说 \" 这题  怎么可能wa\" 这种话. 这样只能起到反效果.

\end{document}

