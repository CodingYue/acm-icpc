\documentclass[a4paper, 11pt, nofonts, nocap, fancyhdr]{ctexart}
\usepackage{graphicx} 

\usepackage[margin=60pt]{geometry}

\setCJKmainfont[BoldFont={FZHei-B01}, ItalicFont={FZKai-Z03}]{FZShuSong-Z01}
\setCJKmonofont{FZShuSong-Z01}

\CTEXoptions[today=small]

\pagestyle{plain}


% \fancyhead[L]{\small{team name}}
% \fancyhead[C]{\small{FSTC 2014 - 05 - 训练报告}}
% \fancyhead[R]{\small{2014年8月2日}}

\renewcommand{\thesubsubsection}{Problem \Alph{subsubsection}.}
\newcommand{\problem}[1]{\subsubsection{#1}}

\title{Fudan ACM-ICPC Summer Training Camp 2014\\第5场训练报告}
\author{Team 1}
\date{\today}

\begin{document}

\maketitle

\section{概况}

本场训练,我们队伍在比赛中完成了8道题目,比赛后完成了2道题目,共完成10道题目。已经完成本场训练所有题目。

\section{训练过程}

(xhm视角)

比赛开始几分钟后才拿到题目,我从前往后看题,lym从后往前看题,yy配环境。过了一会看到C题是裸矩阵,遂上去写,提交RE,苦苦查错时纳指导表示数据配错了才导致RE的,rejudge AC, 20min C题通过(1Y)。

换lym写F,一道并查集的题目,lym写完感觉有点问题,print之后下来思考,yy上去写A题,几分钟后lym改完F题,提交AC,47min F题通过(1Y)。

yy写完了A题,1Y,50min A题通过(1Y)。lym上去写B题,在花了10分钟输入完巨大的样例发现没问题之后提交AC,86min B通过(1Y)。

我上去写D题,提交1WA,迅速修改eps后AC,109min D通过(2Y)。

然后我继续写I题,一个二分+单调性dp的题目,写完调过样例AC,139min I通过(1Y)。

yuege上去写一个裸的三维几何题,G题,写完TLE,发现是复杂度写傻了,改完AC,198min G通过(2Y)。

此时剩下三个题目都已理论AC,J恶心,H傻傻想不清楚,于是我选择了相对难写但是写了肯定能过的E题,花了1小时才写完,写完瞬间过样例!提交wa,遂火速打印换lym写H。下机仔细看了看发现建图的时候有点小问题,连了几条20亿的边权,一加就爆了,修改几次后AC,290min E题通过(4Y)。然后紧急调H题,最后3分钟我找到了一个trick,lym火速改,可惜有一个地方\&\&打成||了..


\section{解题报告}

\problem{A. Ginkgo Numbers}

\begin{description}
\item[负责] 杨越
\item[情况] 比赛时通过 - 50min - 1Y
\end{description}

拿题目所给两个条件写写即可.

\problem{B. Stylish}

\begin{description}
\item[负责] 刘炎明
\item[情况] 比赛时通过 - 86min - 1Y
\end{description}

题目怎么说怎么写,

\problem{C. One-Dimensional Cellular Automaton}

\begin{description}
\item[负责] 邢皓明
\item[情况] 比赛时通过 - 20min - 1Y
\end{description}

裸矩阵乘法。

\problem{D. Find the Outlier}

\begin{description}
\item[负责] 邢皓明
\item[情况] 比赛时通过 - 109min - 2Y
\end{description}

枚举哪个是错的,看剩下的$d+2$个能不能用d次多项式拟合,由于d很小直接高斯消元插值即可。
注意$eps$,$1e-4$是wa,改成$1e-3$就AC了。

\problem{E. Sliding Block Puzzle}

\begin{description}
\item[负责] 邢皓明
\item[情况] 比赛时通过 - 290min - 4Y
\end{description}

状态:国王在什么位置,两个空格在国王旁边的哪个方位。
状态只有$4 \times n \times n\leq10000$个。
转移注意到搬箱子和搬空格是等价的,相当于两个空格移动到所需位置,用最短路预处理转移,总复杂度$O(n^4 + SPFA(N^2,N^2))$。

\problem{F. Never Wait for Weights}

\begin{description}
\item[负责] 刘炎明
\item[情况] 比赛时通过 - 47min - 1Y
\end{description}

边权并查集,除了father\[\]数组之外,额外记录一个offest\[\]表示这个点到父亲的边权,路径压缩时顺便维护即可,这样就能支持查询了。

\problem{G. Let There Be Light}

\begin{description}
\item[负责] 杨越
\item[情况] 比赛时通过 - 198min - 2Y
\end{description}

枚举光源的集合, 然后算哪些需要被删除.
简单的计算几何.
注意预处理,不然30s时限也保不住。

\problem{H. Company Organization}

\begin{description}
\item[负责] 邢皓明、刘炎明、杨越
\item[情况] 赛后通过
\end{description}

先二分答案.
对于a包含于b的关系, 建一条b->a的边.
a=b可以看做一条双向边.
操作三用来判断是否有冲突. 如果a b处于一个强联通分量里, 那么操作三是不合法的.
之后对于所有交集为空的操作, a, b. 记A集合为从a出发能dfs到的点, B集合为从b出发能dfs到的点. A 交 B 必然都为空集. 对于属于A不属于B的集合记为C, 属于B不属于A的集合记为D. 那么暴力枚举C, D的元素, 他们两两之间的交集一定为空. 
最后用操作5直接判断是否可行.
注意空集不等于空集.

\problem{I. Beautiful Spacing}

\begin{description}
\item[负责] 邢皓明
\item[情况] 比赛时通过 - 139min - 1Y
\end{description}

先二分答案, 然后直接dp, 决策区间存在单调性. 

\problem{J. Cubic Colonies}

\begin{description}
\item[负责] 杨越
\item[情况] 赛后通过
\end{description}

考虑一个$1\times 12$的格子, 对于每个点向每个点连边, 对于每条边存在交点. 一定是一些等比分点. 于是对于每个合法的面, 处理出每条边的$2..9$等比分点, 然后一个面之间的点连边. 最后跑一次最短路即可。

\section{总结}

最后8题,本来能9题的。
鄙队有无法决杀光环。

\end{document}
