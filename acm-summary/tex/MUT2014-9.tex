\documentclass[a4paper, 11pt, nofonts, nocap, fancyhdr]{ctexart}
\usepackage{graphicx} 

\usepackage[margin=60pt]{geometry}

\setCJKmainfont[BoldFont={FZHei-B01}, ItalicFont={FZKai-Z03}]{FZShuSong-Z01}
\setCJKmonofont{FZShuSong-Z01}

\CTEXoptions[today=small]

\pagestyle{plain}



% \fancyhead[L]{\small{team name}}
% \fancyhead[C]{\small{FSTC 2014 - 05 - 训练报告}}
% \fancyhead[R]{\small{2014年8月2日}}

\renewcommand{\thesubsubsection}{Problem \Alph{subsubsection}.}
\newcommand{\problem}[1]{\subsubsection{#1}}

\title{Fudan ACM-ICPC Summer Training Camp 2014\\第N场训练报告}
\author{Team 1}
\date{\today}

\begin{document}

\maketitle

\section{概况}

本场训练,我们队伍在比赛中完成了9道题目,比赛后完成了2道题目,共完成11道题目。已经完成本场训练所有题目。


\section{训练过程}

(yy视角)

由于今天非常奇葩, 只有一份题面, 大家把题目拆了开始看题.

我们上机的时候已经有队伍把K题过了, 于是lym去做K题.

[K - 19min - 1Y]

lym本想继续写题, 最后觉得有些问题, 就让xhm写题了. 我由于把F题的N看成$10^4$, 苦苦思考. 然后lym把J题扔给了我, 我就去做J了.

邢皓明写完过了.

[B - 53min - 1Y]

接着lym上去拍I, 拍完过了.

[I - 58min - 1Y]

xhm上去做J, 然后wa了下来看代码, 我推出了J的式子, 上去写了半分钟交发现wa了. 然后下来检查式子, lym开始写E.

5分钟后我发现有个地方写错了, 改了过了.

[J - 88min - 2Y]

然后看了一眼板, 发现大家纷纷过了F题, 于是重新看了看题, 发现$N \leq 1000$, 这时lym写完了E.

[E - 107min - 1Y]

我上去写F, 5分钟写完过了.

[F - 112min - 1Y]

lym帮助xhm查代码, xhm开始写D. 我思考G的做法.

lym查出了xhm代码数组下标可能为负的情况, 改了之后过了A.

[A - 138min - 3Y]

这时我想出了G的正确做法, 由于板子这题需要的板子是lym的, 让lym上去写. 我开始做C题.

[G - 172min - 1Y]

xhm交了一发G, 发现TLE.

于是我开始写C. 一开始写的时候没有思考清楚, 写一半觉得有些恶心, 然后队友们认为$N^2$能过. 于是我迅速码了一个$N^2$的做法上去. 但是TLE了.

于是我下来重新思考$Nlog(N)$的做法, xhm上去优化D题的代码.

D题又提交了一发, 但是还是TLE.

于是我开始写C. 想清楚了之后写了10分钟, 但是wa了. 下来思考, 让xhm继续调题.

最终发现有个地方手贱了.

[C - 279min - 3Y]

D题由于是一开始没有想清楚, 所以导致修改的时候改的不是很正确, 到最后没有调出来.

\section{解题报告}

\problem{Another OCD Patient}

\begin{description}
\item[负责] 邢皓明
\item[情况] 比赛中通过 - 138min(3Y)
\end{description}

我是A题的报告。

\problem{Boring Sum}

\begin{description}
\item[负责] 邢皓明
\item[情况] 比赛中通过 - 53min(1Y)
\end{description}

我是B题的报告。

\problem{Closed Paths}

\begin{description}
\item[负责] 杨越
\item[情况] 比赛中通过 - 279min(3Y)
\end{description}

我是C题的报告。

\problem{Dividing a String}

\begin{description}
\item[负责] 邢皓明
\item[情况] 赛后通过
\end{description}

我是D题的报告。

\problem{Emmet}

\begin{description}
\item[负责] 刘炎明
\item[情况] 比赛中通过 - 107min(1Y)
\end{description}

这种表达式解析的题目随便递归一下就做出来了,最多写个15min,调个5min。看版上大家不敢写其实令人很无语。

\problem{Fast Matrix Calculation}

\begin{description}
\item[负责] 杨越
\item[情况] 比赛中通过 - 122min(1Y)
\end{description}

我是F题的报告。

\problem{GGS-DDU}

\begin{description}
\item[负责] 杨越、刘炎明
\item[情况] 比赛中通过 - 178min(1Y)
\end{description}

挺裸的一个最小树形图模型,按题目所说连边,再从高级向低级连回边即可。使用$O(|V||E|)$的做法已经足够通过。

\problem{Handling the Past}

\begin{description}
\item[负责] 刘炎明
\item[情况] 赛后通过
\end{description}

看了一个小时一直以为是真·Retroactive Data Structures,结果发现其实不是,本题所有Timestamp都给定,并且允许离线,于是从一个挺麻烦的维护变成了一个傻逼题。

按Timestamp离散化后,用线段树维护+1 -1(push/pop)序列上的区间和及最大后缀和,之后做类似于爬线段树的操作,
先查询右子区间(当然要在$[0,timestamp]$这一段上,下同)的最大后缀和,如果>t(t的含义见后文),说明左子区间没用,答案产生在右子区间,向右走即可。
否则,答案产生在左子区间,这时,求出右子区间的和,若为负数则代表最终pop了左子区间里push进去的值,因此向左走,并从查询栈顶向上第t个元素改为查询第$t + (-sum_r)$个即可。

\problem{Improving the GPA}

\begin{description}
\item[负责] 刘炎明
\item[情况] 比赛中通过 - 58min(1Y)
\end{description}

简单的背包问题。

\problem{Just a Joke}

\begin{description}
\item[负责] 杨越
\item[情况] 比赛中通过 - 88min(2Y)
\end{description}

我是J题的报告。

\problem{Killing Monsters}

\begin{description}
\item[负责] 刘炎明
\item[情况] 比赛中通过 - 19min(1Y)
\end{description}

容易求出怪物走在每个格子的时候受到的伤害(一边扫描即可),然后还用说吗?

\section{总结}

% 谁来就开场罚时问题写一下吧T_T
我是总结。

\end{document}

