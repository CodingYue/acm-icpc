\documentclass[a4paper, 11pt, nofonts, nocap, fancyhdr]{ctexart}
\usepackage{graphicx} 

\usepackage[margin=60pt]{geometry}

\setCJKmainfont[BoldFont={FZHei-B01}, ItalicFont={FZKai-Z03}]{FZShuSong-Z01}
\setCJKmonofont{FZShuSong-Z01}

\CTEXoptions[today=small]

\pagestyle{plain}



% \fancyhead[L]{\small{team 1}}
% \fancyhead[C]{\small{FSTC 2014 - 05 - 训练报告}}
% \fancyhead[R]{\small{2014年8月2日}}

\renewcommand{\thesubsubsection}{Problem \Alph{subsubsection}.}
\newcommand{\problem}[1]{\subsubsection{#1}}

\title{Fudan ACM-ICPC Summer Training Camp 2014\\第13场训练报告}
\author{Team 1}
\date{\today}

\begin{document}

\maketitle

\section{概况}

本场训练,我们队伍在比赛中完成了7道题目,比赛后完成了?道题目,共完成?道题目。已经(?)完成本场训练至少完成8题的要求(或,已经完成本场训练所有题目)。

这里是其他概况。

\section{训练过程}

(XHM视角)

一上来lym配环境,我和yy胡乱看题。看到A就不淡定了,尼玛写了3遍的题,怒上机拍之。

(可我似乎忘了之前虽然做过三遍但是每次都wa了两三发。。

yy说D题他做过,暴力最大流就能过,所以第一次提交wa掉之后,lym上机写D的裸网络流。

[problem D - 69min - 1Y]

[problem A - 74min - 4Y]

通过A题之后下机和越哥讨论题,越哥说有一道贪心不记得怎么做了,是八中上的题,问我记不记得。

我一看。。尼玛想多了吧。。排序题,遂让lym上去写。

[problem F - 84min - 1Y]

I题是个简单的dp,状态表示很显然,写就可以了,同样由主代码手lym完成。

[problem I - 107min - 2Y]

C题描述很像以前见过的一道神题,仔细读了之后发现不是那道= =只是一道简单的polya计数,yuege第一发快速幂打错了- -|||

[problem C - 157min - 2Y]

这时候突然想起来还有一道K题,于是我上去写,因为前几天刚做过一道几乎一样的,很快写完后犯了数组开小/没清数组两个手贱,改正后AC。

[problem K - 210min - 3Y]

在我和yuege调试期间lym写完了E题,一直没过样例,调试无果,我过去仔细一看。。尼玛模板打错了。。p和0印在纸上太像了233,改完就过样例了,提交AC。

[problem E - 216min - 1Y]

然后lym表示又累又饿去睡一会,我和yuege搞J题,期间忘了仔细跟yuege确认做法,结果越哥耿直的拍了个$n^2$,赶紧被我拉下来了,跟他说了我的做法后yuege很气愤:你不早说!于是吵吵闹闹的最后没有写完$O(nlogn)$的做法。

\section{解题报告}


\problem{Peach Blossom Spring}

\begin{description}
\item[负责] 负责一、负责二
\item[情况] 比赛中通过 - 233min(1Y)
\end{description}

我是A题的报告。

\problem{Astrolabe}

\begin{description}
\item[负责] 刘炎明
\item[情况] 赛前通过
\end{description}

第一部分嘛,题目怎么说怎么写。第二部分嘛,这个问题显然至少是NP-Hard的,点数又只有12,因此我们直接爆搜就可以了,记得加入最优性剪枝。

\problem{Zhuge Liang's Stone Sentinel Maze}

\begin{description}
\item[负责] 负责一、负责二
\item[情况] 比赛后通过
\end{description}

我是C题的报告。

\problem{Island Transport}

\begin{description}
\item[负责] 杨越
\item[情况] 比赛时通过 - 69min
\end{description}

平面图网络流是一个经典问题,可以通过转化为对偶图再跑最短路来解决,可惜这题数据出的不好,直接暴力网络流就可以了。

\problem{GRE Words}

\begin{description}
\item[负责] 刘炎明
\item[情况] 比赛时通过 - 216min
\end{description}

做一遍后缀数组,每次搞出连续的后缀区间,在线段树上查询一下即可。修改的时候直接暴力修改进去。

\problem{Buildings}

\begin{description}
\item[负责] 邢皓明
\item[情况] 比赛时通过 - 69min
\end{description}

直接按a+b排序即可,好像是NOIP题?

\problem{Dynamic Lover}

\begin{description}
\item[负责] 呵呵
\item[情况] 尚未通过
\end{description}

写都不想写。


\problem{Multiple}

\begin{description}
\item[负责] 负责一、负责二
\item[情况] 尚未通过
\end{description}

我是H题的报告。

\problem{4 substrings problem}

\begin{description}
\item[负责] 刘炎明
\item[情况] 比赛时通过
\end{description}

很显然的一个DP,状态记录哪些串已经用过,已经开始的cover还能cover多久即可。


\problem{Raining}

\begin{description}
\item[负责] 负责一、负责二
\item[情况] 尚未通过
\end{description}

我是J题的报告。



\problem{Graph}

\begin{description}
\item[负责] 负责一、负责二
\item[情况] 尚未通过
\end{description}

我是K题的报告。

\section{总结}

今天虽然看起来踩了其他队伍好多题,但实际上感觉如果是regional的话,节奏这么慢肯定拿不到奖杯的,最后J题没有通过也是一个致命伤,但这也跟蔽队没有吃东西有关。。如果好好打的话大概能8题少罚时吧,一定几率能冲9题。

\end{document}
